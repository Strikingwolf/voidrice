\newcommand{\e}{\varepsilon}
\newcommand{\de}{\delta}
\renewcommand*{\d}{\mathop{}\!\mathrm{d}}
\newcommand{\tri}{\overset{\triangle}{\leq}}
\newcommand{\unif}{\xrightarrow{unif.}}
\newcommand{\ptwise}{\xrightarrow{ptwise.}}
\newcommand{\ball}[2]{\mathcal{B}_{#1}\left(#2\right)}
\DeclareMathOperator{\osc}{osc}
\DeclareMathOperator{\diam}{diam}

\pgfdeclarelayer{left}
\pgfdeclarelayer{right}
\pgfdeclarelayer{plot}
\pgfsetlayers{main, right, left, plot}
\tikzset{
  pics/riemann sum/.style args={#1:#2:#3}{
    code={
      \pgfmathsetmacro{\leftpoint}{#1}
      \pgfmathsetmacro{\rightpoint}{#2}
      \pgfmathsetmacro{\movecount}{#3}
      \pgfmathsetmacro{\step}{(\rightpoint-\leftpoint)/\movecount}
      \begin{scope}[local bounding box=riemann]
      \foreach \i [count=\c from 0] in {1,...,\movecount} {
        \pgfonlayer{left}
        \path[riemann sum/left sum] (\leftpoint+\c*\step, {temp(\leftpoint+\c*\step)}) rectangle (\leftpoint+\i*\step, 0);
        \endpgfonlayer
        \pgfonlayer{right}
        \path[riemann sum/right sum] (\rightpoint-\c*\step, {temp(\rightpoint-\c*\step)}) rectangle (\rightpoint-\i*\step, 0);
        \endpgfonlayer
      }
      \pgfonlayer{plot}
      \draw [domain=#1:#2, riemann sum/riemann line] plot (\x, {temp(\x)});
      \endpgfonlayer
      \end{scope}
      \draw[->, riemann sum/riemann axis] ([xshift=-5mm]riemann.west) -- ([xshift=5mm]riemann.east);
      \draw[->, riemann sum/riemann axis] ([yshift=-5mm]riemann.south) -- ([yshift=5mm]riemann.north);
    }
  },
  riemann sum/.search also=/tikz,
  riemann sum/.cd,
  function/.style 2 args={declare function={temp(#1)=#2;}},
  left sum/.style={draw},
  right sum/.style={draw},
  riemann line/.style={},
  riemann axis/.style={},
  left/.style={left sum/.append style={#1}},
  right/.style={right sum/.append style={#1}},
  line/.style={riemann line/.append style={#1}},
  axis/.style={riemann axis/.append style={#1}}
}
\newcommand{\riemannsum}[2][-1:1:2]{\pic[riemann sum/.cd,#2] {riemann sum=#1};}

\pgfplotsset{compat=newest}
\pgfplotsset{
    integral segments/.code={\pgfmathsetmacro\integralsegments{#1}},
    integral segments=3,
    integral samples/.code={\edef\integralsamples{#1}},
    integral samples = 10,
    integral min/.style args={#1:#2}{
        ybar interval,
        domain=#1:#2,
        samples=\integralsegments+1,
        x filter/.code={
                               \edef\lastx{\pgfmathresult}
                                \pgfmathresult
                              },%
        y filter/.code={%
                          \pgfmathparse{(#2/(\integralsegments))/\integralsamples}%
                           \edef\tempstep{\pgfmathresult}%
                            \pgfmathparse{f(\lastx)}%
                            \edef\tempa{\pgfmathresult}%
                            \edef\tempb{\pgfmathresult}%
                            \foreach \x in {0,1,...,\integralsamples}%
                               {%
                                \pgfmathparse{f(\lastx+\x*\tempstep)}%
                                 \xdef\tempb{\tempb,\pgfmathresult}%
                                 }%
                               \pgfmathmin{\tempb}{\tempb}
                             },
    },
    integral max/.style args={#1:#2}{
        ybar interval,
        domain=#1:#2,
        samples=\integralsegments+1,
        x filter/.code={
                               \edef\lastx{\pgfmathresult}
                                \pgfmathresult
                              },%
        y filter/.code={%
                          \pgfmathparse{(#2/(\integralsegments))/\integralsamples}%
                           \edef\tempstep{\pgfmathresult}%
                            \pgfmathparse{f(\lastx)}%
                            \edef\tempa{\pgfmathresult}%
                            \edef\tempb{\pgfmathresult}%
                            \foreach \x in {0,1,...,\integralsamples}%
                               {%
                                \pgfmathparse{f(\lastx+\x*\tempstep)}%
                                 \xdef\tempb{\tempb,\pgfmathresult}%
                                 }%
                               \pgfmathmax{\tempb}{\tempb}
                             },
    },
}
\makeatletter
%see https://tex.stackexchange.com/questions/9722/why-am-i-getting-the-pgf-math-error-unknown-function-getargs
\def\pgfmathmax#1#2{%
%   \pgfmathparse{getargs(#1,#2)}%
     \pgfmathparse{#1,#2}%
    \expandafter\pgfmathmax@\expandafter{\pgfmathresult}%
}
\def\pgfmathmin#1#2{%
    \pgfmathparse{#1,#2}%
    \expandafter\pgfmathmin@\expandafter{\pgfmathresult}%
}
